\documentclass[10pt, letterpaper]{article}

% Packages:
\usepackage[
    ignoreheadfoot, % set margins without considering header and footer
    top=2 cm, % seperation between body and page edge from the top
    bottom=2 cm, % seperation between body and page edge from the bottom
    left=2 cm, % seperation between body and page edge from the left
    right=2 cm, % seperation between body and page edge from the right
    footskip=1.0 cm, % seperation between body and footer
    % showframe % for debugging
]{geometry} % for adjusting page geometry
\usepackage{titlesec} % for customizing section titles
\usepackage{tabularx} % for making tables with fixed width columns
\usepackage{array} % tabularx requires this
\usepackage[dvipsnames]{xcolor} % for coloring text
\definecolor{primaryColor}{RGB}{0, 0, 0} % define primary color
\usepackage{enumitem} % for customizing lists
\usepackage{fontawesome5} % for using icons
\usepackage{amsmath} % for math
\usepackage[
    pdftitle={Rajat Sandeep's CV},
    pdfauthor={Rajat Sandeep},
    pdfcreator={LaTeX with RenderCV},
    colorlinks=true,
    urlcolor=primaryColor
]{hyperref} % for links, metadata and bookmarks
\usepackage[pscoord]{eso-pic} % for floating text on the page
\usepackage{calc} % for calculating lengths
\usepackage{bookmark} % for bookmarks
\usepackage{lastpage} % for getting the total number of pages
\usepackage{changepage} % for one column entries (adjustwidth environment)
\usepackage{paracol} % for two and three column entries
\usepackage{ifthen} % for conditional statements
\usepackage{needspace} % for avoiding page brake right after the section title
\usepackage{iftex} % check if engine is pdflatex, xetex or luatex

% Ensure that generate pdf is machine readable/ATS parsable:
\ifPDFTeX
    \input{glyphtounicode}
    \pdfgentounicode=1
    \usepackage[T1]{fontenc}
    \usepackage[utf8]{inputenc}
    \usepackage{lmodern}
\fi

\usepackage{charter}

% Some settings:
\raggedright
\AtBeginEnvironment{adjustwidth}{\partopsep0pt} % remove space before adjustwidth environment
\pagestyle{empty} % no header or footer
\setcounter{secnumdepth}{0} % no section numbering
\setlength{\parindent}{0pt} % no indentation
\setlength{\topskip}{0pt} % no top skip
\setlength{\columnsep}{0.15cm} % set column seperation
\pagenumbering{gobble} % no page numbering

\titleformat{\section}{\needspace{4\baselineskip}\bfseries\large}{}{0pt}{}[\vspace{1pt}\titlerule]

\titlespacing{\section}{
    % left space:
    -1pt
}{
    % top space:
    0.3 cm
}{
    % bottom space:
    0.2 cm
} % section title spacing

\renewcommand\labelitemi{$\vcenter{\hbox{\small$\bullet$}}$} % custom bullet points
\newenvironment{highlights}{
    \begin{itemize}[
        topsep=0.10 cm,
        parsep=0.10 cm,
        partopsep=0pt,
        itemsep=0pt,
        leftmargin=0 cm + 10pt
    ]
}{
    \end{itemize}
} % new environment for highlights


\newenvironment{highlightsforbulletentries}{
    \begin{itemize}[
        topsep=0.10 cm,
        parsep=0.10 cm,
        partopsep=0pt,
        itemsep=0pt,
        leftmargin=10pt
    ]
}{
    \end{itemize}
} % new environment for highlights for bullet entries

\newenvironment{onecolentry}{
    \begin{adjustwidth}{
        0 cm + 0.00001 cm
    }{
        0 cm + 0.00001 cm
    }
}{
    \end{adjustwidth}
} % new environment for one column entries

\newenvironment{twocolentry}[2][]{
    \onecolentry
    \def\secondColumn{#2}
    \setcolumnwidth{\fill, 4.5 cm}
    \begin{paracol}{2}
}{
    \switchcolumn \raggedleft \secondColumn
    \end{paracol}
    \endonecolentry
} % new environment for two column entries

\newenvironment{threecolentry}[3][]{
    \onecolentry
    \def\thirdColumn{#3}
    \setcolumnwidth{, \fill, 4.5 cm}
    \begin{paracol}{3}
    {\raggedright #2} \switchcolumn
}{
    \switchcolumn \raggedleft \thirdColumn
    \end{paracol}
    \endonecolentry
} % new environment for three column entries

\newenvironment{header}{
    \setlength{\topsep}{0pt}\par\kern\topsep\centering\linespread{1.5}
}{
    \par\kern\topsep
} % new environment for the header

\newcommand{\placelastupdatedtext}{% \placetextbox{<horizontal pos>}{<vertical pos>}{<stuff>}
  \AddToShipoutPictureFG*{% Add <stuff> to current page foreground
    \put(
        \LenToUnit{\paperwidth-2 cm-0 cm+0.05cm},
        \LenToUnit{\paperheight-1.0 cm}
    ){\vtop{{\null}\makebox[0pt][c]{
        \small\color{gray}\textit{Last updated in July 2024}\hspace{\widthof{Last updated in July 2024}}
    }}}%
  }%
}%

% save the original href command in a new command:
\let\hrefWithoutArrow\href

% new command for external links:


\begin{document}
\newcommand{\AND}{\unskip
	\cleaders\copy\ANDbox\hskip\wd\ANDbox
	\ignorespaces
}
\newsavebox\ANDbox
\sbox\ANDbox{$|$}

\begin{header}
	\fontsize{25 pt}{25 pt}\selectfont Rajat Sandeep Sen

	\vspace{5 pt}

	\normalsize
	\mbox{Thodupuzha, Kerala}%
	\kern 5.0 pt%
	\AND%
	\kern 5.0 pt%
	\mbox{\hrefWithoutArrow{mailto:rajatsandeepsen@gmail.com}{rajatsandeepsen@gmail.com}}%
	\kern 5.0 pt%
	\AND%
	\kern 5.0 pt%
	\mbox{\hrefWithoutArrow{tel:+919846101882}{+91 9846101882}}%
	\kern 5.0 pt%
	\AND%
	\kern 5.0 pt%
	\mbox{\hrefWithoutArrow{https://rajat.club/}{https://rajat.club}}%
	\kern 5.0 pt%
	\AND%
	\kern 5.0 pt%
	\mbox{\hrefWithoutArrow{https://linkedin.com/in/rajatsandeepsen}{linkedin.com/in/rajatsandeepsen}}%
	\kern 5.0 pt%
	\AND%
	\kern 5.0 pt%
	\mbox{\hrefWithoutArrow{https://github.com/rajatsandeepsen}{github.com/rajatsandeepsen}}%
\end{header}

\vspace{5 pt - 0.3 cm}

\section{Introduction}

\begin{onecolentry}
	Self-taught Full Stack Developer, DevOps, AI Engineer, and FOSS enthusiast
\end{onecolentry}

\vspace{0.2 cm}

\begin{onecolentry}
	Freelancer and Open Source Contributor. Lead FOSS, Tech \& Entrepreneurship communities in campus. Recently Co-Founded startup as CTO.
\end{onecolentry}

% \vspace{0.2 cm}

% \begin{onecolentry}
% 	Committed to continuous learning and applying modern technologies to solve real-world problems through part-time freelancing and open-source contributions as FOSS enthusiast.
% \end{onecolentry}

% \section{Quick Introduction}

% \begin{onecolentry}
%     \begin{highlightsforbulletentries}

%         \item Each section title is arbitrary, and each section contains a list of entries.

%         \item There are 7 unique entry types: \textit{BulletEntry}, \textit{TextEntry},
%         \textit{EducationEntry}, \textit{ExperienceEntry}, \textit{NormalEntry},
%         \textit{PublicationEntry}, and \textit{OneLineEntry}.

%         \item Select a section title, pick an entry type, and start writing your section!

%         \item \href{https://docs.rendercv.com/user_guide/}{Here}, you can find a comprehensive user guide for RenderCV.

%     \end{highlightsforbulletentries}
% \end{onecolentry}

\section{Education}

\begin{twocolentry}{
		Nov 2021 – June 2025
	}
	\textbf{St Joseph's College of Engineering \& Technology, Palai}\end{twocolentry}

\vspace{0.10 cm}
\begin{onecolentry}
	\begin{highlights}
		\item BTech in Artificial Intelligence \& Data Science
		\item CGPA: 8.11 % (\href{https://example.com}{Transcript})
		\item Minor Degree in Robotics
		\item \textbf{Coursework:} Artificial Intelligence, Computer Network, OOPs, DSA, Machine Learning, Deep Learning, NLP
	\end{highlights}
\end{onecolentry}

% \vspace{0.2 cm}

% \begin{twocolentry}{
%         Jan 2019 – Jan 2021
%     }
%     \textbf{St. Augustine's Higher Secondary School Karimkunnam}, Higher Secondary, Computer Science\end{twocolentry}

% \vspace{0.10 cm}
% \begin{onecolentry}
%     \begin{highlights}
%         \item Grade: 95.0\% % (\href{https://example.com}{Transcript})
%         \item \textbf{Coursework:} Basic of C++, Python, HTML, CSS, PHP
%     \end{highlights}
% \end{onecolentry}

\section{Work Experience}

\begin{twocolentry}{
		Jan 2025 – Present
	}
	\textbf{Chief Technical Officer}, Manolo Pvt Ltd -- Palai, Kottayam\end{twocolentry}

\vspace{0.10 cm}
\begin{onecolentry}
	\begin{highlights}
		\item Co-Founded the startup as parent holding company
		\item Currently building \& maintaining infrastructure and backend of 2 SaaS products
	\end{highlights}
\end{onecolentry}

\vspace{0.2 cm}

\begin{twocolentry}{
		May 2025 – May 2025
	}
	\textbf{GenAI Developer Intern}, Beckn Protocol -- FIDE, Bangalore\end{twocolentry}

\vspace{0.10 cm}
\begin{onecolentry}
	\begin{highlights}
		\item Published packages for Digital Public Infrastructure Protocol, Beckn in Typescript, Python \& Go-Lang
		\item Built user friendly agentic server for new Indian Digital Energy Grid project
	\end{highlights}
\end{onecolentry}

\vspace{0.2 cm}

\begin{twocolentry}{
		July 2023 – Dec 2023
	}
	\textbf{Frontend Developer Intern}, Gtech-Mulearn -- TechnoPark, Thrivandrum\end{twocolentry}

\vspace{0.10 cm}
\begin{onecolentry}
	\begin{highlights}
		\item Maintain websites as Remote frontend intern for Mulearn, a edu-tech community
		in India
		\item 6 months experience with Reactjs (TypeScript) and Django (Python)
		\item Rewrote the UI components with better abstraction and reusability
	\end{highlights}
\end{onecolentry}

\vspace{0.2 cm}

\section{Volenteering Experience}

\begin{twocolentry}{
		Jun 2022 – Feb 2025
	}
	\textbf{Chief Technical Officer}, IEDC -- Startup Bootcamp, SJCET Palai\end{twocolentry}

\vspace{0.10 cm}
\begin{onecolentry}
	\begin{highlights}
		\item Promoted to CTO from technical officer in 2024 for leading the tech team
		\item Created \& maintained IEDC website at 2022
		\item Volunteered in organizing IEDC Summit 2022
		\item Actively engaged in organizing hackathons, workshops and tech talks for
		promoting entrepreneurship at campus
	\end{highlights}
\end{onecolentry}

\vspace{0.2 cm}

\begin{twocolentry}{
		Sep 2023 – Dec 2024
	}
	\textbf{FOSS Club Lead}, The Nexus Project -- SJCET Palai\end{twocolentry}

\vspace{0.10 cm}
\begin{onecolentry}
	\begin{highlights}
		\item Restarted the FOSS community at campus
		\item Conducted workshops and events to promote FOSS culture
	\end{highlights}
\end{onecolentry}

\vspace{0.2 cm}

\begin{twocolentry}{
		Jan 2025 – Feb 2025
	}
	\textbf{Designer \& Web Developer}, Tedx -- SJCET Palai\end{twocolentry}

\vspace{0.10 cm}
\begin{onecolentry}
	\begin{highlights}
		\item Helped in organizing TEDx event at campus
		\item Designed poster \& created TEDx website at 2025
	\end{highlights}
\end{onecolentry}

\vspace{0.2 cm}

\begin{twocolentry}{
		Sep 2022 – Jul 2024
	}
	\textbf{TensorFlow Lead}, Google Students Developer Club -- SJCET Palai\end{twocolentry}

\vspace{0.10 cm}
\begin{onecolentry}
	\begin{highlights}
		\item Promoted to TensorFlow Lead in GSDC in 2024
		\item Conducted workshops on TensorFlow, Firebase, Google Cloud Platform
		\item Created \& maintained GDSC website at 2022 \& 2023
	\end{highlights}
\end{onecolentry}

\vspace{0.2 cm}

\begin{twocolentry}{
		Feb 2023 – Dec 2023
	}
	\textbf{Co-Technical Lead}, UiPath Academic Alliance Student -- SJCET Palai\end{twocolentry}

\vspace{0.10 cm}
\begin{onecolentry}
	\begin{highlights}
		\item Joined as Co-Technical lead in 2023 and taught RPA tools
		\item Traveled around kerala and conducted many workshops on UI Path tools
	\end{highlights}
\end{onecolentry}

\vspace{0.2 cm}

\begin{twocolentry}{
		June 2022 – June 2023
	}
	\textbf{Web Developer}, IEEE -- SJCET, Palai\end{twocolentry}

\vspace{0.10 cm}
\begin{onecolentry}
	\begin{highlights}
		\item Developed and maintained IEEE website and organized many events
		\item Volunteered in IEEE Kochi Hub event 2022
	\end{highlights}
\end{onecolentry}

\section{Projects}

\begin{twocolentry}{
		\href{https://github.com/rajatsandeepsen/beckn-typescript}{npmjs/beckn-typescript}
	}
	\textbf{Beckn Packages}\end{twocolentry}

\vspace{0.10 cm}
\begin{onecolentry}
	\begin{highlights}
		\item Developed a set of packages for Beckn protocol in TypeScript, Python \& Go-Lang
		to improve developer experience
		\item Tech Stack: TypeScript, Python, Go-Lang
	\end{highlights}
\end{onecolentry}

\vspace{0.2 cm}

\begin{twocolentry}{
		\href{https://github.com/Milansuman/Helios-Browser}{github/helios-browser}
	}
	\textbf{Helios AI Browser}\end{twocolentry}

\vspace{0.10 cm}
\begin{onecolentry}
	\begin{highlights}
		\item Developed an open source alternative of ARC browser with integrated AI tools
		\item I Integrated GenAI into browser at India FOSS Hackathon
		\item Tech Stack: TypeScript, QT, C++, AI-SDK, Chromium, React
	\end{highlights}
\end{onecolentry}

\vspace{0.2 cm}

\begin{twocolentry}{
		\href{https://github.com/rajatsandeepsen/radar}{github/radar}
	}
	\textbf{Radar - AI Surveillance Engine}\end{twocolentry}

\vspace{0.10 cm}
\begin{onecolentry}
	\begin{highlights}
		\item Developed a major project for college, a web-app and a radar system that can
		detect objects in real-time using YOLOv8 and OpenCV
		\item The system can be used for various applications such as security, surveillance,
		and traffic monitoring
		\item Tech Stack: TypeScript, Python, FastAPI, OpenCV, Tkinder, React, Supabase,
		Yolo-v8
	\end{highlights}
\end{onecolentry}

\vspace{0.2 cm}

\begin{twocolentry}{
		\href{https://asthra.sjcetpalai.ac.in}{asthra.sjcetpalai.ac.in}
	}
	\textbf{Asthra - Tech Fest Website}\end{twocolentry}

\vspace{0.10 cm}
\begin{onecolentry}
	\begin{highlights}
		\item Developed a robotic 3D interactive website for the national level technical
		fest, with dashboard for event management, registration and ticketing.
		\item Tech Stack: Nextjs, Supabase, TailwindCSS, ThreeJS, Framer Motion, tRPC, Redis,
		Cloudflare
	\end{highlights}
\end{onecolentry}

\vspace{0.2 cm}

\begin{twocolentry}{
		\href{https://luttapi.vercel.app/}{luttapi.vercel.app}
	}
	\textbf{LuttAPI - Open source alternative to v0}\end{twocolentry}

\vspace{0.10 cm}
\begin{onecolentry}
	\begin{highlights}
		\item Developed a SaaS for instantly generating \& deploying frontend UI with API
		endpoints or JSON
		\item Tech Stack: Nextjs, AI-SDK, Redis (Upstash), TailwindCSS, Framer Motion, tRPC,
		Github Actions, Anthropic Claude
	\end{highlights}
\end{onecolentry}

\vspace{0.2 cm}

\begin{twocolentry}{
		\href{https://github.com/nexus-sjcet/filesphere.ai}{nexus-sjcet/filesphere.ai}
	}
	\textbf{FileSphere - AI for Drive \& File Management}\end{twocolentry}

\vspace{0.10 cm}
\begin{onecolentry}
	\begin{highlights}
		\item At a hackathon, developed an Action Model AI file management system that can
		categorize, search, and manage files based on user request by itself
		\item Tech Stack: Nextjs, LangChain, GO, tRPC, Llama3
	\end{highlights}
\end{onecolentry}
\vspace{0.2 cm}

\begin{twocolentry}{
		\href{https://www.npmjs.com/package/initiative}{npmjs/initiative}
	}
	\textbf{Initiative - Vercel AI-SDK Extension}\end{twocolentry}

\vspace{0.10 cm}
\begin{onecolentry}
	\begin{highlights}
		\item Developed and published library on NPM that turns Large Languages Models into
		Action Models for sequencial function calling tool chain
		\item Tech Stack: LangChain, Zod, TypeScript, AI-SDK
	\end{highlights}
\end{onecolentry}
\vspace{0.2 cm}

\begin{twocolentry}{
		\href{https://github.com/nexus-sjcet/mindease}{nexus-sjcet/mindease}
	}
	\textbf{MindEase - LAM}\end{twocolentry}

\vspace{0.10 cm}
\begin{onecolentry}
	\begin{highlights}
		\item Developed at Hackathon, a personal whatsapp bot with Action Model AI that can
		performas health care assistances
		\item Tech Stack: Python, LangChain, Prisma, FastAPI, Mistral AI
	\end{highlights}
\end{onecolentry}
\vspace{0.2 cm}

\begin{twocolentry}{
		\href{https://github.com/nexus-sjcet/voice-it}{nexus-sjcet/voice-it}
	}
	\textbf{MultiMedia Commenting Vscode Extension}\end{twocolentry}

\vspace{0.10 cm}
\begin{onecolentry}
	\begin{highlights}
		\item Developed at Hackathon, a vscode extension that allows multi-media commenting
		(img, voice, gif etc) inside any code file
		\item Tech Stack: TypeScript, Vscode API
	\end{highlights}
\end{onecolentry}
\vspace{0.2 cm}

\begin{twocolentry}{
		\href{https://github.com/rajatsandeepsen/scriptw}{rajatsandeepsen/scriptw}
	}
	\textbf{Scriptw - REPL Notebook for Javascript}\end{twocolentry}

\vspace{0.10 cm}
\begin{onecolentry}
	\begin{highlights}
		\item Developed in-browser Jupyter Notebook alternative for Javascript
		\item Tech Stack: JavaScript, CodeMirror, Nextjs, Prisma
	\end{highlights}
\end{onecolentry}
\vspace{0.2 cm}

\begin{twocolentry}{
		\href{https://github.com/rajatsandeepsen/transactions}{rajatsandeepsen/transactions}
	}
	\textbf{Transaction - AI enabled Banking App}\end{twocolentry}

\vspace{0.10 cm}
\begin{onecolentry}
	\begin{highlights}
		\item Developed hands free banking app that can be controlled using Voice as
		Engineering Mini-Project
		\item Tech Stack: Nextjs, LangChain, Supabase, tRPC, Llama3, DrizzleORM, Zod
	\end{highlights}
\end{onecolentry}
\vspace{0.2 cm}

\begin{twocolentry}{
		\href{https://magicui.design/}{magicui.design}
	}
	\textbf{Magic-UI CLI}\end{twocolentry}

\vspace{0.10 cm}
\begin{onecolentry}
	\begin{highlights}
		\item Developed a registry \& CLI for open source project MagicUI (library for Design
		Engineers)
		\item Tech Stack: Nextjs, Zod, Commander, TypeScript, TailwindCSS, Framer-Motion
	\end{highlights}
\end{onecolentry}

\section{Additional Achievements And Awards}

\vspace{0.2 cm}
\begin{onecolentry}
	\textbf{\#BuildWithAI International Hackathon (2025):} Secured Fifth prize by building Multi-Model AI for online news
\end{onecolentry}

\vspace{0.2 cm}
\begin{onecolentry}
	\textbf{Satwa National Level Hackathon (2024):} Secured First Runner-Up by building Action Model AI for health care assistance
\end{onecolentry}

\vspace{0.2 cm}

\begin{onecolentry}
	\textbf{Beyond-the-Loop Hackathon (2024):} Secured First place by building Drive.AI, a Google Drive file system with AI capabilities
\end{onecolentry}

\section{Publications}

\begin{samepage}
	\begin{twocolentry}{
			Dec 2024
		}
		\textbf{Simple Action Model: Enabling LLM to Sequential Function Calling Tool Chain}
	\end{twocolentry}

	\vspace{0.10 cm}

	\begin{onecolentry}
		% \mbox{\textbf{\textit{Rajat Sandeep}}} 

		\vspace{0.10 cm}

		\href{https://ieeexplore.ieee.org/document/10893677}{IEEE Xplore - 10893677}
	\end{onecolentry}
\end{samepage}

\section{Technologies}

\begin{onecolentry}
	\textbf{Languages:} TypeScript, Python, SQL, C++, Java, JavaScript, HTML, CSS
\end{onecolentry}

\vspace{0.2 cm}

\begin{onecolentry}
	\textbf{Frameworks:} Nextjs, Expo, TensorFlow, Sklearn, OpenCV, Tkinder
\end{onecolentry}
\vspace{0.2 cm}

\begin{onecolentry}
	\textbf{Libraries:} Reactjs, React-Native, Expressjs, TailwindCSS, LangChain, DrizzleORM, Prisma, tRPC, Better-Auth, Hono, FastAPI, Pandas etc
\end{onecolentry}
\vspace{0.2 cm}

\begin{onecolentry}
	\textbf{Softwares:} Docker, Bun, Nodejs, PostgreSQL, Nodejs, Redis, Kubernetes, UI Path %, OpenTofu
\end{onecolentry}
\vspace{0.2 cm}

\begin{onecolentry}
	\textbf{Developer Tools:} Zed, Vscode, Git, Github, Figma, Hoppschotch, Ollama, WSL
\end{onecolentry}

\end{document}
